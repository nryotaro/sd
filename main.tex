\documentclass{jlreq}
\usepackage[bookmarks=true,bookmarksnumbered=true,
pdftitle={システム設計},
pdfauthor={nryotaro},
pdfkeywords={System Design; Interview},
pdflang=ja-JP
]{hyperref}
\usepackage{main}
\addbibresource{main.bib}
\begin{document}
\title{システム設計}
\author{nryotaro}
\date{\today}
\maketitle
\tableofcontents
\section{面接の流れ}
\begin{refsection}
  システム全体のアーキテクチャを問われるときは、Propose a Design, Talk about trade-offs, Discuss design choices, Limitationsの4部構成で話す\cite{lc-high}。
  \printjbibliography
\end{refsection}
\section{モノリスではなくマイクロサービスで設計する}
\begin{refsection}
  課題はモノリスで設計するには複雑すぎるので、マイクロサービスで設計する\cite{lc-aa}。
  \subsection{設問}
  \begin{enumerate}[label=問題\arabic*,labelsep=10pt,leftmargin=*]
  \item モノリスと比べたときのマイクロサービスの利点と欠点をのべよ。\label{exe:monolith-micro}
  \end{enumerate}
  \subsection{解答}
  \subsubsection*{\ref{exe:monolith-micro}}
  例えば、技術を適材適所で使いわけられるが、API呼びだしによるレイテンシーが発生する\cite{lc-aa}。
  \printjbibliography
\end{refsection}
\end{document}