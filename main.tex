\documentclass{jlreq}
\usepackage[bookmarks=true,bookmarksnumbered=true,
pdftitle={システム設計},
pdfauthor={nryotaro},
pdfkeywords={System Design; Interview},
pdflang=ja-JP
]{hyperref}
\usepackage{main}
\addbibresource{main.bib}
\begin{document}
\title{システム設計}
\author{nryotaro}
\date{\today}
\maketitle
\tableofcontents
\part{基本}
\begin{section-bib}{面接の流れ}
  システム全体のアーキテクチャを問われるときは、Propose a Design, Talk about trade-offs, Discuss design choices, Limitationsの4部構成で話す\cite{lc-high}。
\end{section-bib}
%
\begin{section-bib}{モノリスではなくマイクロサービスで設計する}
  % モノリスとマイクロサービスの比較を書く。
  課題はモノリスで設計するには複雑すぎるので、マイクロサービスで設計する\cite{lc-aa}。
  \subsection{設問}
  \begin{exercise}
  \item モノリスと比べたときのマイクロサービスの利点と欠点をのべよ。\label{exe:monolith-micro}  
  \end{exercise}
  \subsection{解説}
  \subsubsection*{\ref{exe:monolith-micro}}
  たとえば、複数の技術を適材適所に使えるが、プロセス内の通信がネットワーク越しのAPI呼び出しになる分のレイテンシーが発生する\cite{lc-aa}。  
\end{section-bib}
%
\begin{section-bib}{Queue}
  複数のサービスへ同期的に通信するトランザクションは、通信を非同期にすれば応答の待機時間を短くできる。
  しかし、対向サービスに障害があれば非同期通信が失敗しても、トランザクションは別のサービスにリクエストを送る。
  サービスをキューを介して通信させることで、対向サービスと一時的につながらない間でも、キューにメッセージを蓄積でき、メッセージの消失を防ぐことができる\cite{lc-isc}。
  \subsection{設問}
  \begin{exercise}
  \item 2つのサービスが直接ではなくキューを介して非同期通信する利点をのべよ。また、利点を満足するためのキューへの要件は何か。\label{exe:async-queue}
  \item 主要なメッセージキューのミドルウェアを挙げ、向き、不向きを比較せよ。\label{exe:queue-compare}
  \end{exercise}
  \subsection{解説}
  \subsubsection*{\ref{exe:async-queue}}
  対向システムの障害時のリクエストをキューに一時的に保存できるため、失敗するリクエストを減らすことができる\cite{lc-isc}。
  キューは、中継対象のシステムより、安定していなけばならない。
  キューでなくロードバランサでも負荷分散を実現できるので、キューの利点として負荷分散を先にあげるのはよくないだろうか。
  \subsubsection*{\ref{exe:queue-compare}}
  Kafka, RabbitMQ, ActiveMQがある\cite{lc-isc}。
\end{section-bib}
\begin{section-bib}{双方向通信}
  ロングポーリングは、タイムアウトまでの時間を長くし、サーバからの応答を待つ。
  このとき、サーバーはクライアウントのタイムアウトを検知できない\cite{sdi}。
  \subsection{設問}
  \begin{exercise}
    \item ロングポーリングとWebSocketの使い分けについて議論せよ。
  \end{exercise}
  
\end{section-bib}
\part{ケーススタディ}
\end{document}